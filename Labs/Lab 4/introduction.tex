\noindent In this lab, you will use a two stage melting model to extract the continental and oceanic crust from the primitive mantle of Earth. This greatly simplified model of \emph{differentiation} is able to reproduce the elemental abundances of crustal rocks and offers quantitative constraints about the history of melt extraction from the mantle.

\medskip

\noindent This simplified history begins with the primitive mantle (often referred to as the \emph{bulk silicate earth}). This geochemical reservoir represents the mantle of Earth (after core formation) before the crust was extracted. In other words, the primitive mantle would be the result of could mix the entire crust back into the present day mantle. You will differentiate Earth in two discrete stages of melt extraction:

\begin{itemize}
	\item Stage 1: A partial melt of the \emph{primitive mantle} is extracted to form the continental crust, leaving behind a mantle that is depleted in highly incompatible elements.
	\item Stage 2: A partial melt of the left behind \emph{depleted mantle} is extracted to form the oceanic crust.
\end{itemize}

\noindent You will use \textbf{Fractional melting} equations to complete these stages of differentiation, and you will directly compare your modeled melts to the average composition of continental crust and mid-ocean ridge basalt (oceanic crust).


\medskip
\noindent The \textbf{partition coefficient}, $D$, describes the ratio of concentrations for an element between a mineral and a melt at equilibrium:

$$D=\frac{C_S}{C_L}$$

\noindent \textbf{Fractional melting} describes a scenario where some fraction of a rock melts and that melt is immediately separated from the rock. The concentration of a trace element in the remaining rock is described by this equation:

$$\frac{C_S}{C_0}=(1-F)^{(D^{-1}-1)}$$

\noindent The concentration of a trace element in an infinitely small fraction of melt, often referred to as the \emph{instantaneous melt}, is described by this equation:

$$\frac{C_L}{C_0}= \frac{(1-F)^{(D^{-1}-1)}}{D}$$

\noindent C is the concentration of a trace element. Subscript L represents the melt phase. Subscript S represents the solid phase. C$_0$ means at the initial conditions when solid is 100\% of the system. D is the partition coefficient, and F is the melt fraction (where 1 is 100\% melt).
