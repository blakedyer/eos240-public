In this lab, you will compare \textbf{Ni} concentrations in olivines to modeled predictions of \textbf{Ni} concentrations during \emph{batch} and \emph{fractional} crystallization. You will find the partition coefficients, $D$, that best fit each crystallization model, and you will compare those results to experimentally determined partitioning for \textbf{Ni} to argue for which crystallization model more accurately describes the magmatic system that these olivine samples formed in. Recall:

\medskip
\noindent The \textbf{partition coefficient}, $D$, describes the ratio of concentrations for an element between a mineral and a melt at equilibrium:

$$D=\frac{C_S}{C_L}$$

\noindent \textbf{Batch crystallization} describes a scenario where some fraction of a magmatic system crystallizes and remains in equilbrium with the liquid.

$$
	\frac{C_S}{C_0}=\frac{D}{F+D(1-F)}
$$

\noindent \textbf{Fractional crystallization} describes the continuous removal of mineral percipitates from a melt:

$$
	\frac{C_S}{C_{0}}=D(F)^{D-1}
$$

\noindent C is the concentration of a trace element. Subscript L represents the melt phase. Subscript S represents the solid phase. C$_0$ means at the initial conditions when melt is 100\% of the system. D is the partition coefficient, and F is the melt fraction (where 1 is 100\% melt).
