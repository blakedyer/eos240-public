In this lab, you will use peridotites and chondritic meteorites to build a geochemical model of Earth's upper mantle.
In the first question, you will compare variations in Mg, Al, and Si in Lherzolite xenoliths to variations in chondritic meteorites to determine a subset of xenoliths that may represent pristine (unmelted) upper mantle.
In question 2, you will compare the average composition of this subset of samples to the volatile rich CI chondrites. To explain some of the chemical differences you will find between Earth's mantle and CI chondrites, you will rely on your understanding of the Goldschmidt element classification scheme and element condensation temperatures.
